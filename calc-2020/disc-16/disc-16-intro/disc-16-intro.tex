% Font size, paper type
\documentclass[12pt]{article}
% Aesthetic margins
\usepackage[margin=1in]{geometry}
% Core math packages,
% Mathtools loads amsmath, and amsmath gives basic math symbs
% Amsfonts & amssymb are misc. symbols you might need
\usepackage{mathtools, amsfonts, amssymb}
% Links in a pdf
\usepackage{hyperref}
% Use in pictures, graphs, and figures
\usepackage{graphicx}
% Header package
\usepackage{fancyhdr}
% Underlining with line breaks
\usepackage{ulem}
% Adjust accordingly given warning messages
\setlength{\headheight}{15pt}
% So we can more easily format text with pictures
\usepackage{float}
% Images and drawing graphs
\usepackage{tikz}

% Sets footer
\pagestyle{fancy}
% Removes default footer style
\fancyhf{}

\rhead{
  Shengdong Li
  Calc 3
}

\rfoot{
  Page \thepage{}
}

% Makes links look more appealing
\hypersetup{
    colorlinks=true,
    linkcolor=blue,
    filecolor=magenta,      
    urlcolor=cyan,
}

% \usepackage{indentfirst}

\begin{document}
\title{Disc 16 Intro}
\author{by Shengdong Li}
\date{21 January 2021}
\maketitle

\section{Setup}

Using Google RNG, the random numbers that I recieved were
\[5, 4, 10, 2\]
Using these numbers, I was able to generate the initial point \(P\left(5, 4\right)\) and terminating point \(Q\left(10, 2\right)\) to create vector \(\vec{PQ}\).

\section{Components of Vector}

\begin{align*}
	\vec{V} & = \begin{bmatrix}10-5 \\ 2-4\\\end{bmatrix} \\
	        & = \begin{bmatrix}5 \\-2\end{bmatrix} \\
	V_x     & = 5                         \\
	V_y     & = -2
\end{align*}

\section{Unit Vector}
To find the unit vector, recall that
\begin{align*}
	V & = \left\lVert V \right\rVert\cdot U    \\
	U & = \frac{V}{\left\lVert V \right\rVert}
\end{align*}
where \(U\) is the unit vector.
\begin{align*}
	\intertext{Therefore, first calculate the magnitude}
	\left\lVert V \right\rVert & = \sqrt{5^ {2}+{(-2)}^ {2}}                    \\
	                           & = \sqrt{25 + 4}
	                           & = \sqrt{29}
	\intertext{Then we can calculate the unit vector by dividing the components}
	U                          & = \frac{1}{\sqrt{29}}\begin{bmatrix}
		5 \\
		-2
	\end{bmatrix} \\
	                           & = \begin{bmatrix}
		\frac{5\sqrt{29}}{29} \\
		\frac{-2\sqrt{29}}{29}
	\end{bmatrix}
\end{align*}


\section{Adding New Vector}
Once again, using Google RNG, the random numbers that I recieved were
\[
	10, 4, 2, 5
\]
Using these numbers, I was then able to generate the initial point \(A\left(10, 4\right)\) and terminating point \(B\left(2, 5\right)\) to create vector \(\vec{AB}\).

Then calculating the components gives
\begin{align*}
	V           & = \begin{bmatrix}
		2 - 10 \\
		5 - 4
	\end{bmatrix}                             \\
	            & = \begin{bmatrix}
		-8 \\
		1
	\end{bmatrix}
	\intertext{Then we can add the components of the two vectors together}
	V_1 + V_2   & = \begin{bmatrix}
		5 \\ -2
	\end{bmatrix} + \begin{bmatrix}
		-8 \\ 1
	\end{bmatrix} \\
	\Aboxed{V_R & = \begin{bmatrix}
			-3 \\ -1
		\end{bmatrix}}
\end{align*}

\section{Creating a Square}
\begin{figure}[H]
	\begin{tikzpicture}[x=0.4cm,y=0.4cm]
		\draw[latex-latex, thin, draw=gray] (-4,0)--(16,0) node [right] {$x$}; % l'axe des abscisses
		\draw[latex-latex, thin, draw=gray] (0,-4)--(0,10) node [above] {$y$}; % l'axe des ordonnées
		\draw [dotted, gray] (-4,-4) grid (16,10);

    \coordinate (P) at (5, 4);
    \coordinate (Q) at (10, 2);
    \coordinate (R) at (7, 9);
    \coordinate (S) at (12, 7);
    
    \foreach \coordinate\n in {(P)/P, (Q)/Q, (R)/R, (S)/S}
    {
      \node[circle,inner sep=1pt,fill=black,label=above:\n] at \coordinate {};
    }

		\draw[->] (P) -- (Q);
		\draw[->] (P) -- (R);
		\draw[->] (Q) -- (S);
		\draw[->] (R) -- (S);
	\end{tikzpicture}
\end{figure}

\end{document}