\documentclass[12pt]{article}
\usepackage[margin=1in]{geometry}
\usepackage{amsfonts, amsmath, amssymb}
\usepackage{hyperref}
\hypersetup{
    colorlinks=true,
    linkcolor=blue,
    filecolor=magenta,      
    urlcolor=cyan,
}

\usepackage{fancyhdr}

\pagestyle{fancy}
\fancyhf{}

\rhead{
  Shengdong Li
  Calc 3
}

\rfoot{
  Page \thepage
}

\usepackage{indentfirst}

\begin{document}
\title{Response to Malaika Shah}
\author{by Shengdong Li}
\date{25 September 2020}
\maketitle

\section{About the equation editor}
Hey Malaika, thanks for replying to my post!

I'm \textit{very} glad that you asked asked me about my equation editor. You see, I made a \href{https://youtu.be/iH12bU1XLrk}{video} on this topic a while back talking about the downsides of the Canvas equation editor and why you should use the fullblown latex language instead. In it, I give a quick introduction to Latex and its semantics and demonstrate my workflow solving an integral using Desmos and a text editor. TL;DR, try \href{https://www.overleaf.com/}{Overleaf} to get a feel for Latex.

The setup I'm currently using is \href{https://code.visualstudio.com/}{VSCode} as my text editor with the Latex Workshop extension, but it requires the installation of a LaTex distribution which can take a lot of time. You might want to try this option if you already have an offline text editor that you like. Here's the \href{https://www.latex-project.org/get/}{download page} if you're interested.

\section{Concering the 4th problem}

I think that the way that you solved the $4$th problem via long dividing the $\frac{x^{3}}{4x^{2}-1}$ and then solving for each individual piece of the resulting $\frac{1}{4}\int_{ }^{ }xdx$ and $\frac{1}{4}\int_{ }^{ }\frac{x}{4x^{2}-1}dx$ is definitely a valid approach to solve the problem.

One statement that you made that I found really interesting was that for you, when you saw that the $x$s are only $1$ power apart you went to long divison. For me, I first saw that the combined values of the $u$ and $du$ values in terms of $x$ had an equal power to the numerator, and so I tried $u$-sub first! I oftentimes forget that long division is always a solid way to solve a problem given an improper fraction, so I think that seeing your comment and thought process is helpful.

Anyways, thanks for replying again, I wish you luck on your journey to find a better alternative to the Canvas equation editor!
\end{document}