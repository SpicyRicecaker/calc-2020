\documentclass[12pt]{article}
\usepackage[margin=1in]{geometry}
\usepackage{amsfonts, amsmath, amssymb}
% \usepackage{hyperref}
% \hypersetup{
%     colorlinks=true,
%     linkcolor=blue,
%     filecolor=magenta,      
%     urlcolor=cyan,
% }

\usepackage{fancyhdr}

\pagestyle{fancy}
\fancyhf{}
\setlength{\headheight}{15pt}

\rhead{
  Shengdong Li
  Calc 3
}

\rfoot{
  Page \thepage
}

\usepackage{indentfirst}

\begin{document}
\title{Response to Brian Chou}
\author{by Shengdong Li}
\date{26 September 2020}
\maketitle

\section{Intro}
Hey Brian, great job with the initial post! The equation editor that you used made it really clear and easy to follow every step of the post.

For both problems number 1 and 2, we both used $u$-sub to solve the problem, but I've noticed that coincidentally, they can both also be solved using partial fraction decomposition. So I was wondering, is partial fraction decomposition good or bad for solving these problems?

\section{Solving $\int_{ }^{ }\frac{dx}{x^{2}+10x+25}$ via decomp}

\begin{align}
  \intertext{First we factor the bottom}
  \int_{ }^{ }\frac{dx}{x^{2}+10x+25} & = \int_{ }^{ }\frac{dx}{\left(x+5\right)^{2}} \\
  \intertext{Then we can define the decomposed fraction form}
  \frac{1}{\left(x+5\right)^{2}}      & =\frac{A}{x+5}+\frac{B}{\left(x+5\right)^{2}} \\
  \intertext{Plugin values to solve}
  1                                   & =A\left(x+5\right)+B                          \\
  x                                   & =-5                                           \\
  1                                   & =0+B                                          \\
  B                                   & =1                                            \\
  x                                   & =0                                            \\
  1                                   & =5A+1                                         \\
  5A                                  & =0                                            \\
  A                                   & =0                                            \\
  \intertext{Plugin solved values to integral (We get the exact same integral we started with)}
                                      & =\int_{ }^{ }\frac{1}{\left(x+5\right)^{2}}dx
  \intertext{Now we have to stop stalling and we're forced to $u$-sub}
  u                                   & =x+5                                          \\
  du                                  & =dx                                           \\
                                      & =\int_{ }^{ }\frac{1}{u^{2}}du                \\
                                      & =\int_{ }^{ }u^{-2}du                         \\
  \intertext{Into integration}
                                      & =\frac{u^{-1}}{-1}+C                          \\
                                      & = \boxed{-\frac{1}{x+5}+C}
\end{align}

\setcounter{equation}{0}

\section{Solving $\int_{ }^{ }\frac{xdx}{x^{2}+10x+25}$ via decomp}

\begin{align}
  \intertext{we first have to factor again}
  \int_{ }^{ }\frac{xdx}{x^{2}+10x+25} & =\int_{ }^{ }\frac{xdx}{\left(x+5\right)^{2}}
  \intertext{then define the decomposed form}
  \frac{x}{\left(x+5\right)^{2}}       & =\frac{A}{x+5}+\frac{B}{\left(x+5\right)^{2}}
  \intertext{then plugin values to solve}
  x                                    & =A\left(x+5\right)+B                                                       \\
  x                                    & =-5                                                                        \\
  B                                    & =-5                                                                        \\
  x                                    & =0                                                                         \\
  0                                    & =5A+B                                                                      \\
  0                                    & =5A+-5                                                                     \\
  5                                    & =5A                                                                        \\
  A                                    & =1                                                                         \\
  \intertext{Now we get a resulting integral that looks a lot like what we would've gotten if we used $u$-sub to split the fraction up in the first place. We can then integrate to solve...}
                                       & =\int_{ }^{ }\frac{1}{x+5}dx-5\int_{ }^{ }\frac{1}{\left(x+5\right)^{2}}dx
  \intertext{Here we can actually just plugin our answer from the previous problem}
                                       & =\ln\left(\left|x+5\right|\right)-5\left(-\frac{1}{x+5}+C\right)           \\
                                       & =\ln\left(\left|x+5\right|\right)+\frac{5}{x+5}-5C                         \\
                                       & =\boxed{\ln\left(\left|x+5\right|\right)+\frac{5}{x+5}+C}
\end{align}

\section{In conclusion...}
In terms of pure complexity, the partial fraction decomposition in both problems added a whole bunch of steps made up of algebra to solve for the terms, so it's a lot less convenient than $u$-sub. 

In the first problem partial fraction decomp didn't even help at all because the problem could not be factored further. However, the process of solving the second problem I found very interesting. First, midway through solving the problem, the resulting $\int_{ }^{ }\frac{1}{x+5}dx-5\int_{ }^{ }\frac{1}{\left(x+5\right)^{2}}dx$ was actually the exact same as the $\int_{ }^{ }\frac{1}{u}du-5\int_{ }^{ }\frac{1}{u^{2}}du$ from $u$-sub. Second, the $\int_{ }^{ }\frac{xdx}{\left(x+5\right)^{2}}$ was the exact same problem and fraction as the first problem. I think that this may give a clue on how rational fractions are broken down?????
\end{document}