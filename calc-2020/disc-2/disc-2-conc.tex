\documentclass[12pt]{article}
\usepackage[margin=1in]{geometry}
\usepackage{amsfonts, amsmath, amssymb}
\usepackage{polynom}
% \usepackage{hyperref}
% \hypersetup{
%     colorlinks=true,
%     linkcolor=blue,
%     filecolor=magenta,      
%     urlcolor=cyan,
% }

\usepackage{fancyhdr}
% \setlength{\headheight}{15pt}

\pagestyle{fancy}
\fancyhf{}

\rhead{
  Shengdong Li
  Calc 3
}

\rfoot{
  Page \thepage
}

\usepackage{indentfirst}

\begin{document}
\title{In Conclusion}
\author{by Shengdong Li}
\date{27 September 2020}
\maketitle

\section{Concerning Problem Set A}
In problem set A, the commonality between every single problem was the $\left(x+5\right)^{2}$ repeated linear factor in the denominator. Also, the first two problems, which both only hand $(x+5)^2$ in the denominator could be solved with $u$-sub very easily, while the third problem, which tacked on an extra $(x-3)$ required partial fraction decomp. This leads me to believe that when there's only one unique factor on the denominator, in a proper fraction, $u$-sub should be considered. 

\section{Concerning Problem Set B}
In problem set B, you could either start the problem with $u$-sub, or you could try partial fraction decomp. In both methods, the rational express ended up being broken into two pieces. In partial fraction decomp, we got $\int_{ }^{ }\frac{1}{x+5}dx-5\int_{ }^{ }\frac{1}{\left(x+5\right)^{2}}dx$, while using $u$-sub, we got $\int_{ }^{ }du+\int_{ }^{ }\frac{1}{u}du$. This reminded me that the goals of $u$-sub and partial fraction decomp in solving the integral of a rational expression is very similar: you're trying to break the expression into simpler pieces.
\end{document}