% Font size, paper type
\documentclass[12pt]{article}
% Aesthetic margins
\usepackage[margin=1in]{geometry}
% Core math packages,
% Mathtools loads amsmath, and amsmath gives basic math symbs
% Amsfonts & amssymb are misc. symbols you might need
\usepackage{mathtools, amsfonts, amssymb}
% Links in a pdf
\usepackage{hyperref}
% Use in pictures, graphs, and figures
\usepackage{graphicx}
% Header package
\usepackage{fancyhdr}
% Underlining with line breaks
\usepackage{ulem}
% Adjust accordingly given warning messages
\setlength{\headheight}{15pt}
% So we can more easily format text with pictures
\usepackage{float}

% Sets footer
\pagestyle{fancy}
% Removes default footer style
\fancyhf{}

\rhead{
  Shengdong Li
  Calc 3
}

\rfoot{
  Page \thepage
}

% Makes links look more appealling
\hypersetup{
    colorlinks=true,
    linkcolor=blue,
    filecolor=magenta,      
    urlcolor=cyan,
}

% \usepackage{indentfirst}

\begin{document}
\title{Disc 15 Intro}
\author{by Shengdong Li}
\date{7 January 2021}
\maketitle

Hello everyone, I was assigned
\[
xy=12
\]
\begin{align*}
\intertext{To get it into parametric form, first put it into slope intercept form}
y&=\frac{12}{x}\\
\intertext{We can replace $\frac{12}{x}$ with $t$, then solve for $x$}
t&=\frac{12}{x}\\
x&=\frac{12}{t}\\
\intertext{As a result}
y&=t\\
\intertext{And we're left with}
  \Aboxed{c\left(t\right)&=\left(\frac{12}{t},\ t\right)}
\intertext{To get it into polar form, we just need to remember}
x&=r\cos\theta\\
y&=r\sin\theta\\
\intertext{Then plugin and solve for $r$}
r^{2}\sin\theta\cos\theta&=12\\
r^{2}&=\frac{12}{\sin\theta\cos\theta}\\
r&=\sqrt{\frac{12}{\sin\theta\cos\theta}}\\
\intertext{Recall that }
2\sin\theta\cos\theta&=\sin2\theta\\
\sin\theta\cos\theta&=\frac{1}{2}\sin2\theta\\
\intertext{Plugin}
r&=\sqrt{\frac{12}{\frac{1}{2}\sin2\theta}}\\
\Aboxed{r&=2\sqrt{6\csc2\theta}}
\end{align*}

\end{document}
