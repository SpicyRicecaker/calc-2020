% Font size, paper type
\documentclass[12pt]{article}
% Aesthetic margins
\usepackage[margin=1in]{geometry}
% Core math packages,
% Mathtools loads amsmath, and amsmath gives basic math symbs
% Amsfonts & amssymb are misc. symbols you might need
\usepackage{mathtools, amsfonts, amssymb}
% Links in a pdf
\usepackage{hyperref}
% Use in pictures, graphs, and figures
\usepackage{graphicx}
% Header package
\usepackage{fancyhdr}
% Underlining with line breaks
\usepackage{ulem}
% Adjust accordingly given warning messages
\setlength{\headheight}{15pt}
% So we can more easily format text with pictures
\usepackage{float}

% Sets footer
\pagestyle{fancy}
% Removes default footer style
\fancyhf{}

\rhead{
  Shengdong Li
  Calc 3
}

\rfoot{
  Page \thepage
}

% Makes links look more appealling
\hypersetup{
    colorlinks=true,
    linkcolor=blue,
    filecolor=magenta,      
    urlcolor=cyan,
}

% \usepackage{indentfirst}

\begin{document}
\title{Response to Malea Cesar}
\author{by Shengdong Li}
\date{8 January 2021}
\maketitle

\begin{align*}
\intertext{Hey Malea! Great job with the initial post (s), I'm really sorry to hear that you did the wrong problem, it's happened to me as well on more than one occasion and I know how bad it feels.}
\intertext{With that being said, I do just have a small comment about you conversion from the cartesian equation to parametric equations. Looking at Rita's graph of your parametric equation, although the shape of the graph matches with the original function, the parametric included a \(\pm\), and therefore had to be represented as two parametric equations}
  c_1\left(t\right)=\left(t,\ \sqrt{2t-t^{2}}\right)\\
  c_2\left(t\right)=\left(t,\ -\sqrt{2t-t^{2}}\right)
\end{align*}
\section{Converting via \(x=r\cos t+h\) and \(y=r\sin t+h\)}
\begin{align*}
\intertext{Because of this, I don't believe the parametric equations counted as a straight conversion. However, this is a really easy thing to fix. Just like how for converting to polar equations you have }
\cos\theta&=\frac{x}{r}\\
\sin\theta&=\frac{y}{r}\\
\intertext{For parametric equations, you have }
r\cos t+h&=x\\
r\sin t+k&=y\\
\intertext{Where \(h\) and \(k\) are the coordinates of the center of the table}
  \intertext{We can clearly see that in}
\left(x-1\right)^{2}+y^{2}&=1\\
  \intertext{The radius of the circle would be}
r^{2}&=1\\
r&=1\\
\intertext{And the center of the circle would then be}
  &\left(1,\ 0\right)\\
h&=1\\
k&=0\\
\intertext{Plugging everything in, we get the equations}
1\cdot\cos t+1&=x\\
x&=\cos t+1\\
1\cdot\sin t+0&=y\\
y&=\sin t\\
\intertext{And therefore the \(c(t)\) would be}
  \Aboxed{c\left(t\right)&=\left(\cos t+1,\ \sin t\right)}
\end{align*}
Graphed, I think that the location that the graph moves makes more sense, it moves counterclockwise as \(t\) increases and the circle fully completes after a full \(\sin\) or \(\cos\) period (please see embed below!)
\section{Checking/Converting back}
\begin{align*}
\intertext{To check and convert back, we could use the trig identity}
\sin^{2}\theta+\cos^{2}\theta&=1\\
\intertext{We know that}
x&=\cos t+1\\
\sin t&=1\\
\intertext{So now we just have to plug back in}
\cos t&=x-1\\
  \Aboxed{\left(x-1\right)^{2}+y^{2}&=1}
\end{align*}
I hoped that this helped!

\bigskip
Cheers,

-Andy

\end{document}
