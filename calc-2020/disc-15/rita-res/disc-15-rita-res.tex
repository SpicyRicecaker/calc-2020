% Font size, paper type
\documentclass[12pt]{article}
% Aesthetic margins
\usepackage[margin=1in]{geometry}
% Core math packages,
% Mathtools loads amsmath, and amsmath gives basic math symbs
% Amsfonts & amssymb are misc. symbols you might need
\usepackage{mathtools, amsfonts, amssymb}
% Links in a pdf
\usepackage{hyperref}
% Use in pictures, graphs, and figures
\usepackage{graphicx}
% Header package
\usepackage{fancyhdr}
% Underlining with line breaks
\usepackage{ulem}
% Adjust accordingly given warning messages
\setlength{\headheight}{15pt}
% So we can more easily format text with pictures
\usepackage{float}

% Sets footer
\pagestyle{fancy}
% Removes default footer style
\fancyhf{}

\rhead{
  Shengdong Li
  Calc 3
}

\rfoot{
  Page \thepage{}
}

% Makes links look more appealing
\hypersetup{
    colorlinks=true,
    linkcolor=blue,
    filecolor=magenta,      
    urlcolor=cyan,
}

% \usepackage{indentfirst}

\begin{document}
\title{Response to Rita Thammakoune}
\author{by Shengdong Li}
\date{9 January 2021}
\maketitle

\begin{align*}
\intertext{Hello Rita! Great job on the initial post. Looking at your reflection and Jonathon's post above, it seems that the polar equation}
\theta&=\frac{\pi}{3}\\
\intertext{Isn't graphable. Looking at the official Desmos \href{https://support.desmos.com/hc/en-us/articles/202595139-Polar-Graphing}{documentation}, it seems that Desmos does not implement this feature yet. Polar equations with no \(r\) are declared as `nonlinear', but  0 is still a linear equation in which \(a=0\)}
\end{align*}
With that being said, Jonathan and Haelin already graphed your equations and talked about this issue, so I wanted to try and convert your polar and parametric equations back into cartesian form and also comment slightly on your method of solving for the parametric equation
\section{Checking Polar}
\begin{align*}
\intertext{Given the polar equation}
\theta&=\frac{\pi}{3}\\
\intertext{We just have to recall that }
\tan\left(\theta\right)&=\frac{y}{x}\\
\intertext{and therefore}
\theta&=\tan^{-1}\left(\frac{y}{x}\right)\\
\intertext{Plugging this in, we get}
\tan^{-1}\left(\frac{y}{x}\right)&=\frac{\pi}{3}\\
\frac{y}{x}&=\tan\left(\frac{\pi}{3}\right)\\
y&=x\sqrt{3}\\
c\left(t\right)&=\left(t,t\sqrt{3}\right)
\end{align*}
\section{Checking Parametric}
\begin{align*}
\intertext{The parametric check is very straightforward because we declared \(x=t\) in the conversion}
x&=t\\
y&=t\sqrt{3}\\
y&=x\sqrt{3}\\
\intertext{I know that you mentioned that the easiest way to convert is to just substitute \(x\) for \(t\)}
x&=t\\
\intertext{But one other way that you could've converted cartesian to polar was to substitute the entire \(x\) term along with the coefficient}
t&=x\sqrt{3}\\
\intertext{This is slightly better because it is usually more representative of a problem. Then we would just convert normally}
x&=\frac{\sqrt{3}}{3}t\\
y&=t\\
\intertext{With the final \(c\left(t\right)\) being}
\Aboxed{c\left(t\right)&=\left(\frac{\sqrt{3}}{3}t,\ t\right)}
\end{align*}

I hoped that this helped!

\bigskip
Cheers,

-Andy

\end{document}
