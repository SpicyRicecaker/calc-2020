% Font size, paper type
\documentclass[12pt]{article}
% Aesthetic margins
\usepackage[margin=1in]{geometry}
% Core math packages,
% Mathtools loads amsmath, and amsmath gives basic math symbs
% Amsfonts & amssymb are misc. symbols you might need
\usepackage{mathtools, amsfonts, amssymb}
% Links in a pdf
\usepackage{hyperref}
% Use in pictures, graphs, and figures
\usepackage{graphicx}
% Header package
\usepackage{fancyhdr}
% Underlining with line breaks
\usepackage{ulem}
% Adjust accordingly given warning messages
\setlength{\headheight}{15pt}
% So we can more easily format text with pictures
\usepackage{float}

% Sets footer
\pagestyle{fancy}
% Removes default footer style
\fancyhf{}

\rhead{
  Shengdong Li
  Calc 3
}

\rfoot{
  Page \thepage{}
}

% Makes links look more appealing
\hypersetup{
    colorlinks=true,
    linkcolor=blue,
    filecolor=magenta,      
    urlcolor=cyan,
}

% \usepackage{indentfirst}

\begin{document}
\title{Response to Rita Thammakoune}
\author{by Shengdong Li}
\date{9 January 2021}
\maketitle

\begin{align*}
\intertext{Hello Haelin and Rita, thank you both so much for responding to my initial post!}
\intertext{Haelin graphed both functions and came up with the same solutions that I did, so thank you for the heads up and verification}
\intertext{Rita, on the other hand, also graphed my polar equation, but solved the parametric equation by setting}
x&=t\\
\intertext{Then plugging it into}
y&=\frac{12}{x}\\
\intertext{To finally arrive at}
c\left(t\right)&=\left(\frac{12}{t}\right)\\
\intertext{Which I think is a perfectly valid parametric equation as well!}
\end{align*}

Again, thank you both for replying to my initial post,

\bigskip

-Andy

\end{document}
