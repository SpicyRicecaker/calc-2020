% Font size, paper type
\documentclass[12pt]{article}
% Aesthetic margins
\usepackage[margin=1in]{geometry}
% Core math packages,
% Mathtools loads amsmath, and amsmath gives basic math symbs
% Amsfonts & amssymb are misc. symbols you might need
\usepackage{mathtools, amsfonts, amssymb}
% Links in a pdf
\usepackage{hyperref}
% Use in pictures, graphs, and figures
\usepackage{graphicx}
% Header package
\usepackage{fancyhdr}
% Underlining with line breaks
\usepackage{ulem}
% Adjust accordingly given warning messages
\setlength{\headheight}{15pt}
% So we can more easily format text with pictures
\usepackage{float}

% Sets footer
\pagestyle{fancy}
% Removes default footer style
\fancyhf{}

\rhead{
  Shengdong Li
  Calc 3
}

\rfoot{
  Page \thepage
}

% Makes links look more appealling
\hypersetup{
    colorlinks=true,
    linkcolor=blue,
    filecolor=magenta,      
    urlcolor=cyan,
}

% \usepackage{indentfirst}

\begin{document}
\title{Response to Westin Boyd}
\author{by Shengdong Li}
\date{20 December 2020}
\maketitle

Hello Westin,

I found it really interesting that you talked in your post about how negative constants can completely change the shape of your graph. 

Negative constants were something that I played around with but never considered in the beginning because I couldn't and still can't figure out the period for Desmos graphs. As a result, I thought that they were super buggy because the graph would keep changing as $t$ increases. 

From your post, I realize now that negative constants are very important to consider! The sharp vertices made me realize that $\left|\frac{a}{b}\right|$ actually determines the number of "vertices" for the graph, which is more important than I realized because at $a=-4$ and $b=1$ we get the asteroid graph! It's so impressive how epicycloid can seem to change into a lot of the other graphs in our discussion.

This also explains why when $a$ is a decimal, like say $a=1.1$, the graph becomes very complicated, because a decimal vertex would shift the graph over with each iteration of $t$.

Thanks, 

Andy
\end{document}