% Font size, paper type
\documentclass[12pt]{article}
% Aesthetic margins
\usepackage[margin=1in]{geometry}
% Core math packages,
% Mathtools loads amsmath, and amsmath gives basic math symbs
% Amsfonts & amssymb are misc. symbols you might need
\usepackage{mathtools, amsfonts, amssymb}
% Links in a pdf
\usepackage{hyperref}
% Use in pictures, graphs, and figures
\usepackage{graphicx}
% Header package
\usepackage{fancyhdr}
% Underlining with line breaks
\usepackage{ulem}
% Adjust accordingly given warning messages
\setlength{\headheight}{15pt}
% So we can more easily format text with pictures
\usepackage{float}

% Sets footer
\pagestyle{fancy}
% Removes default footer style
\fancyhf{}

\rhead{
  Shengdong Li
  Calc 3
}

\rfoot{
  Page \thepage
}

% Makes links look more appealling
\hypersetup{
    colorlinks=true,
    linkcolor=blue,
    filecolor=magenta,      
    urlcolor=cyan,
}

% \usepackage{indentfirst}

\begin{document}
\title{Response to Deyvik Bhan}
\author{by Shengdong Li}
\date{20 December 2020}
\maketitle

Hey Deyvik, 
Thank you so much for responding to my initial post!

When I was first playing around with initial figures on desmos to post my initial post, I noticed that the graph would change to this bean-shape figure, but I wasn't quite sure what the figure was.

Thanks for telling me that it was a cardiod, and also proving that when substituting the same $a$ and $b$ into the equation of the epicycloid, you could arrive at the equation of the cardoid through algebra.

Perhaps when the graph of the epicycloid looks like other figures, such as the asteroid, one could derive the formula of it as well.

Again, thanks for replying to my initial post!
- Andy
\end{document}