% Font size, paper type
\documentclass[12pt]{article}
% Aesthetic margins
\usepackage[margin=1in]{geometry}
% Core math packages,
% Mathtools loads amsmath, and amsmath gives basic math symbs
% Amsfonts & amssymb are misc. symbols you might need
\usepackage{mathtools, amsfonts, amssymb}
% Links in a pdf
\usepackage{hyperref}
% Use in pictures, graphs, and figures
\usepackage{graphicx}
% Header package
\usepackage{fancyhdr}
% Underlining with line breaks
\usepackage{ulem}
% Adjust accordingly given warning messages
\setlength{\headheight}{15pt}
% So we can more easily format text with pictures
\usepackage{float}

% Sets footer
\pagestyle{fancy}
% Removes default footer style
\fancyhf{}

\rhead{
  Shengdong Li
  Calc 3
}

\rfoot{
  Page \thepage
}

% Makes links look more appealling
\hypersetup{
    colorlinks=true,
    linkcolor=blue,
    filecolor=magenta,      
    urlcolor=cyan,
}

% \usepackage{indentfirst}

\begin{document}
\title{Response to Adithi Mahankali}
\author{by Shengdong Li}
\date{17 December 2020}
\maketitle

Hello Adithi, 

Great job with the initial post! It was very clear and concise, and you posted really early as well.

As I'm sure you know, I had the epicyloid parametric equation, and building off on your reply to my initial post, I'd like to affirm again that scaling of the asteroid graph is directly proportional to $a$, and I think that at this point we've shown that parametric equations with $\sin$ and $\cos$ scale almost exactly the same as you would expect a circle to. 

Furthermore, with $a$ being $0$ the graph also didn't exist for the epicycloid. Even with the extra $b$ constant, when $a$ equalled $0$, the $b$ actually cancelled out the equation to $0$. 

Another thing that I noticed about the asteroid graph is that the $t$ required to draw the entire star is the period of the $\sin$ and $\cos$ functions. With the default parametric at $a=1$
$$
\left(\cos^{3}t,\ \sin^{3}t\right)
$$
the star is completed when $t=2\pi$.
However, if we add an extra $\frac{1}{4}$ to the inside of the sinusoids, the parametric becomes
$$
\left(\cos^{3}\frac{t}{4},\ \sin^{3}\frac{t}{4}\right)
$$
so the period of $\cos$ and $\sin$ then becomes
$$
\frac{2\pi}{\frac{1}{4}}=8\pi
$$
and the asteroid completes at 
$$
t=8\pi
$$
Compared to the epicyloid, 
$$
x(t)=\left(a+b\right)\cos t-b\cos\left(\left(\frac{a}{b}+1\right)t\right)
$$
and
$$
y(t)=\left(a+b\right)\sin t-b\sin\left(\left(\frac{a}{b}+1\right)t\right)
$$
it had an $a$ and $b$ value, but actually, the $a$ value didn't seem to have any impact on when the function completed, it only seemed like the $b$ value was based off of the period, where
$$
T=\frac{2\pi}{\frac{1}{b}}
$$
You can take a look at the Desmos graph \href{https://www.desmos.com/calculator/1obgsminu6}{here}

That's what I noticed, great job on the initial post Adithi!

\end{document}