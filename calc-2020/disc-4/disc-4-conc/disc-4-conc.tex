\documentclass[12pt]{article}
\usepackage[margin=1in]{geometry}
\usepackage{amsfonts, amsmath, amssymb}
\usepackage{hyperref}
\usepackage{mathtools}
\hypersetup{
    colorlinks=true,
    linkcolor=blue,
    filecolor=magenta,      
    urlcolor=cyan,
}
\usepackage{graphicx}
\usepackage{fancyhdr}
\setlength{\headheight}{15pt}

\pagestyle{fancy}
\fancyhf{}

\rhead{
  Shengdong Li
  Calc 3
}

\rfoot{
  Page \thepage
}

% \usepackage{indentfirst}

\begin{document}
\title{In Conclusion}
\author{by Shengdong Li}
\date{11 October 2020}
\maketitle

\section{Intro}
This discussion taught me a lot about the different ways to solve a differential equation using formulas and seperation of variables. I also discovered some new tips and tricks in to use while solving using seperation of variables.

\section{Omar Nassar}
From Omar, I realized that a formula such as
$$
A=M\left(\frac{Ce\left(\frac{k}{\sqrt{M}}\right)^{t}-1}{Ce\left(\frac{k}{\sqrt{M}}\right)^{t}+1}\right)
$$
can be used for differential equations in the form of 
$$
A'=k\sqrt{A}\left(1-\frac{A}{M}\right)
$$
\section{Michael Yan}
From Michael, I learned how to get to the aforementioned formula using a handy $u$-sub trick. This involves replacing the $A$ in $A'=k\sqrt{A}\left(1-\frac{A}{M}\right)$ with $u^{2}$, then solving the resulting $\left(u^{2}\right)'$ in $\left(u^{2}\right)'=k\sqrt{u^{2}}\left(1-\frac{u^{2}}{M}\right)$ with chain rule. This made me realize even more how useful and widely applicable $u$-subbing can be, and for solving differential equations of different forms I'll always have this technique in my back pocket.
\section{Ethan Durham}
Finally, from Ethan's reply to my comment on Hanxun's post, I reaffirmed my understanding of the constant $C$ value's ability to absorb calculations. However, I also realized how useful it was to take out the negative in $\ln\left(P\right)-\ln\left(A-P\right)=kt+C$ to get $\ln\left(A-P\right)-\ln\left(P\right)=C-kt$, since it makes solving for the end equation, and the end result itself, look a lot prettier with much less work.
\end{document}