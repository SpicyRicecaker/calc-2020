\documentclass[12pt]{article}
\usepackage[margin=1in]{geometry}
\usepackage{amsfonts, amsmath, amssymb}
\usepackage{polynom}
\usepackage{hyperref}
\usepackage{mathtools}
\hypersetup{
    colorlinks=true,
    linkcolor=blue,
    filecolor=magenta,      
    urlcolor=cyan,
}
\usepackage{graphicx}

\usepackage{fancyhdr}
\setlength{\headheight}{15pt}
\usepackage{graphicx}


\pagestyle{fancy}
\fancyhf{}

\rhead{
  Shengdong Li
  Calc 3
}

\rfoot{
  Page \thepage
}

\usepackage{indentfirst}

\begin{document}
\title{Initial Post}
\author{by Shengdong Li}
\date{7 October 2020}
\maketitle

\section{Rogawski 10.3 Problem 13}

A tissue culture grows until it has a maximum area of $M\:cm^2$. The area $A(t)$ of the culture at time $t$ may be modeled by the differential equation $A'=k\sqrt{A}\left(1-\frac{A}{M}\right)$ where $K$ is a growth constant.


Let A(t) be the area at time $t$ (hours) of a growing tissue culture with initial size $A(0) = 1\:cm^2$ , assuming that the maximum area is $M = 16\:cm^2$ and the growth constant
is $k = 0.1$.

\bigskip

(a) Find a formula for $A(t)$. Note: The initial condition is satisfied for
two values of the constant $C$. Choose the value of $C$ for which A(t) is
increasing.

(b) Determine the area of the culture at $t = 10$ hours.

\section{Summary of struggles}

I was pluggin' and chuggin' along using the $y=\frac{L}{1+Ce^{-kt}}$ formula until I ran into this problem, which I couldn't put into the standard form of $\frac{dy}{dt}=ky\left(1-\frac{y}{L}\right)$ due to its $\sqrt{A}$.

The textbook mentioned (in the question above) something about setting $A=u^2$, but I had no idea how to actually implement that, so I proceeded to try and solve it using seperation by parts. However, the steps were absolutely atrocious and as I arrived at the answer I began to wonder how this question was even remotely legal for a homework problem.

\bigskip

I would love to see someone's work in solving this problem!
\end{document}