\documentclass[12pt]{article}
\usepackage[margin=1in]{geometry}
\usepackage{amsfonts, amsmath, amssymb}
\usepackage{hyperref}
\usepackage{mathtools}
\hypersetup{
    colorlinks=true,
    linkcolor=blue,
    filecolor=magenta,      
    urlcolor=cyan,
}
\usepackage{graphicx}
\usepackage{fancyhdr}
\setlength{\headheight}{15pt}

\pagestyle{fancy}
\fancyhf{}

\rhead{
  Shengdong Li
  Calc 3
}

\rfoot{
  Page \thepage
}

% \usepackage{indentfirst}

\begin{document}
\title{Response to Ethan Durham}
\author{by Shengdong Li}
\date{11 October 2020}
\maketitle

Hello Ethan!
Thanks for replying to my comment to Hanxun. My answer of $\frac{2000}{1-\frac{1}{Ce^{0.6t}}}$ definitely did not exactly match up with the textbook's answer of $\frac{2000}{1-Ce^{-0.6t}}$. However, I disagree with your comment that I made a mistake due to missing a negative in the absolute values.

In actuality, I left my comment slightly off from the end answer because I thought it was obvious how to manipulate the expression to the textbook answer. I apologize to you and to Hanxun about the confusion that it might've caused.

Below, I'll go through how to exactly match my answer with the textbook's, but also exactly why our answers differ. I think that you'll find that the cause is a lot more simple than the $\pm$ caused by absolute values and natural logs.

\section{Manipulating \texorpdfstring{$\frac{2000}{1-\frac{1}{Ce^{0.6t}}}$}{Lg}}

\begin{align}
  \intertext{First, we seperate the denominator}
  \frac{2000}{1-\frac{1}{Ce^{0.6t}}} & =\frac{2000}{1-\frac{1}{C}\cdot\frac{1}{e^{0.6t}}} \\
  \intertext{Next, the denominator can also be written as a negative exponent}
                                     & =\frac{2000}{1-\frac{1}{C}\cdot e^{-0.6t}}         \\
  \intertext{Literally whatever calculation you do to $C$ that does not include a variable does not matter. You can have $C^{1111111111111111111111}$ and it would still end up as $C$. Therefore, it is only natural that $\frac{1}{C}$ becomes $C$}
  \Aboxed{                           & =\frac{2000}{1-Ce^{-0.6t}} }
  \intertext{The above answer matches the textbook, but to further illustrate the point about $C$, you can absorb the negative.}
                                     & =\frac{2000}{1+Ce^{-0.6t}}                         \\
\end{align}
\subsection{Why is this important?}
I don't think that you choose $-C$ because the absolute value in the $\ln$ allows it. You choose it because the $C$ value itself allows it.

\setcounter{equation}{0}

\section{Why do our answers differ?}
I think that in your post you explained it:

\begin{align}
  \ln\left(P\right)-\ln\left(A-P\right)=kt+C \\
  \ln\left(A-P\right)-\ln\left(P\right)=C-kt
\end{align}

``After attempting to integrate this multiple times by hand, I've realized the best next step from here is to multiply both sides by ``-1'' for two reasons. The first reason is because I know that I am trying to match the ``Logistic Function model''. The second reason is because this will set us up for a cleaner fraction down the line.''

I think that your comment here is very smart because this is something that I completely missed when I solved Hanxun's post for the first time, when I multiplied the negative into the following expression at the point
$$
  -\left(\ln\left(P-2000\right)-\ln\left(P\right)\right)
$$
\setcounter{equation}{0}
Below, I'd like to show the steps in continuing solving Hanxun's problem, but following your advice!
\subsection{Solving from \texorpdfstring{$-\left(\ln\left(P-2000\right)-\ln\left(P\right)\right)=0.6t+C$}{Lg}}
\begin{align}
  -\left(\ln\left(P-2000\right)-\ln\left(P\right)\right) & =0.6t+C                    \\\
  -\ln\left(1-\frac{2000}{P}\right)                      & =0.6t+C                    \\
  \ln\left(1-\frac{2000}{P}\right)                       & =C-0.6t                    \\
  1-\frac{2000}{P}                                       & =Ce^{-0.6t}                \\
  -\frac{2000}{P}                                        & =Ce^{-0.6t}-1              \\
  \frac{2000}{P}                                         & =1-Ce^{-0.6t}              \\
  \Aboxed{P                                                      & =\frac{2000}{1-Ce^{-0.6t}}}
\end{align}
\setcounter{equation}{0}
\section{Proof that you can manipulate the \texorpdfstring{$C$}{Lg}}
Finally, I'll show you that whatever you do to the $C$ doesn't matter because it resolves itself when you solve for it.
\begin{align}
P&=\frac{2000}{1-\frac{1}{C}e^{-0.6t}}\\
500&=\frac{2000}{1-\frac{1}{C}}\\
500\left(1-\frac{1}{C}\right)&=2000\\
1-\frac{1}{C}&=4\\
\frac{1}{C}&=-3\\
C&=-\frac{1}{3}\\
P&=\frac{2000}{1-\frac{1}{-\frac{1}{3}}e^{-0.6t}}\\
\Aboxed{P&=\frac{2000}{1+3e^{-0.6t}}}
\end{align}
And this is the same equation that you got using the formula.
\section{Conclusion}
Thank you so much for replying to my post. While I did want to correct you on your mistake in pointing out my mistake, I also want to thank you for showing me that pulling out the negative at the point of $\ln\left(P\right)-\ln\left(A-P\right)=kt+C$ makes the rest of the equation much easier to solve.
\paragraph{}
Cheers,
Andy
\end{document}