\documentclass[12pt]{article}
\usepackage[margin=1in]{geometry}
\usepackage{amsfonts, amsmath, amssymb}
\usepackage{polynom}
\usepackage{hyperref}
\usepackage{mathtools}
\hypersetup{
    colorlinks=true,
    linkcolor=blue,
    filecolor=magenta,      
    urlcolor=cyan,
}
\usepackage{graphicx}

\usepackage{fancyhdr}
\setlength{\headheight}{15pt}

\pagestyle{fancy}
\fancyhf{}

\rhead{
  Shengdong Li
  Calc 3
}

\rfoot{
  Page \thepage
}

\usepackage{indentfirst}

\begin{document}
\title{Response to Joshua Ji}
\author{by Shengdong Li}
\date{3 October 2020}
\maketitle

\section{Intro}

Hello Joshua,
I also did my initial post on population growth in the United States as well. We even had the same population rate of change of $0.6\%$! One question that I've always wondered about is how long it would take for the population of the United States to reach 1 billion. Using your particular below, I'll try to solve for that.

\section{Time until 1 billion}

\begin{align}
  \intertext{Given the particular solution...}
  P                                            & =327200000e^{0.006t}                                        \\
  \intertext{We set $P$ equal to 1 billion}
  1000000000                                   & =327200000e^{0.006t}                                        \\
  \frac{1000000000}{327200000}                 & =e^{0.006t}                                                 \\
  \ln\left(\frac{1000000000}{327200000}\right) & =0.006t                                                     \\
  t                                            & =\frac{\ln\left(\frac{1000000000}{327200000}\right)}{0.006} \\
  &\approx\boxed{186.2}\text{ years}
\end{align}

\section{Parsing the data}
Since your particular solution was built using the point $(0, 327200000)$, I was a little bit confused as to what year the population was based on. Based off of \href{https://www.desmos.com/calculator/8abrp6yuwb}{Google's Public Data} I found the population in 2018 to be around 327.2 million people. This means that 186 years from 2018 would be at 2204.

\section{Conclusion}
Since I was just a little kid, I never thought that the United state's population wouldn't reach 1 billion. However, through solving for the theoretical number of years until we reach 1 billion through your particular solution, combined with the fact that the United State's population growth rate is decreasing, I now realize that it won't be a likely prospect our lifeimte...
\end{document}