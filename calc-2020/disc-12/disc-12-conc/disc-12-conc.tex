% Font size, paper type
\documentclass[12pt]{article}
% Aesthetic margins
\usepackage[margin=1in]{geometry}
% Core math packages,
% Mathtools loads amsmath, and amsmath gives basic math symbs
% Amsfonts & amssymb are misc. symbols you might need
\usepackage{mathtools, amsfonts, amssymb}
% Links in a pdf
\usepackage{hyperref}
% Use in pictures, graphs, and figures
\usepackage{graphicx}
% Header package
\usepackage{fancyhdr}
% Underlining with line breaks
\usepackage{ulem}
% Adjust accordingly given warning messages
\setlength{\headheight}{15pt}
% So we can more easily format text with pictures
\usepackage{float}

% Sets footer
\pagestyle{fancy}
% Removes default footer style
\fancyhf{}

\rhead{
  Shengdong Li
  Calc 3
}

\rfoot{
  Page \thepage
}

% Makes links look more appealling
\hypersetup{
    colorlinks=true,
    linkcolor=blue,
    filecolor=magenta,      
    urlcolor=cyan,
}

% \usepackage{indentfirst}

\begin{document}
\title{In Conclusion}
\author{by Shengdong Li}
\date{6 December 2020}
\maketitle

While in this week's discussion, I did manage to solve my initial problem of 
$$
  xe^{-x^{3}}
$$
it was not before messing up and accidentally solving William's problem of
$$
  \cos(x^2)
$$
first.

So while doing double the workload really sucked, I think that it was still practice and allowed me to discern more differences between Taylor Series and Riemann Sums.

\section{Taylor Series}
From my observations of Taylor Series, I noticed a few qualities.
\begin{itemize}
  \item From my comment to William comparing the differences of the different degree of Taylor polynomials for $\cos(x)$, I found out that after a certain number of terms are added to the taylor polynomial to calculate a bound, it becomes very accurate. From $n=7$ onwards, the area seemed to not change at all up to the $10$th decimal place.
  \item However, in the beginning it seems to be very innaccurate, with the difference of the fourth degree polynomial with the later degree polynomials being more than $.004$
  \item Over a long bound or if there are a lot of bends within the bounds of a function, Taylor can become exponentially innaccurate. I noticed this in the graph of $\cos(x)$ compared to its $4$th degree Taylor polynomial, which kept going downwards instead of up. In a comment to my post, Katherine also agreed that more bends could make Taylor less accurate
\end{itemize}

\section{Riemann Sums}
\begin{itemize}
  \item Riemann Sums is much much easier to calculate than Taylor series, requiring only basic algebra.
  \item It's also much more accurate than Taylor with less strips/step size vs. terms in the Taylor polynomial. Katherine's calculation of doubling the $n$ value or the number of strips in Riemann sums only led to a difference in the thousandths at $0.005$.
  \item However, continuing with the previous point it becomes less good compared to Taylor with many more terms, as Taylor is accurate up to the $10$th decimal place whereas Taylor still keeps changing its $8$th decimal place at $n=20$.
  \item If not using Simpson's rule, the difference is rather large. In Williams comment to my initial post, he found that using trapezoid rule led to a difference greater than $0.1$!
\end{itemize}

From these observations and analyses about Taylor series and Riemann sums, I came to the conclusion that to get the general idea of the area of a function, use Riemann Sums. However, if one wanted to know the specific part of a function very well, Taylor should be used instead.

\section{Exact value}
With the use cases of Taylor Series and Riemann Sums in mind, to get the exact value of
$$
  \int_{0}^{1}xe^{-x^{3}}dx
$$
down to the thousandths, I decided to add terms to the Taylor polynomial until it did not change at the thousandths place any more.

This occured at $n=5$, and I found that
$$
  \int_{0}^{1}\left(\sum_{n=0}^{5}\frac{\left(-1\right)^{n}\cdot x^{3n+1}}{n!}\right)dx = 0.3498\ldots
$$
which, rounded to the nearest thousandths is $\boxed{0.350}$.
\end{document}