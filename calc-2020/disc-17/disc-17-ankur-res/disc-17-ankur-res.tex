% Font size, paper type
\documentclass[12pt]{article}
% Aesthetic margins
\usepackage[margin=1in]{geometry}
% Core math packages,
% Mathtools loads amsmath, and amsmath gives basic math symbs
% Amsfonts & amssymb are misc. symbols you might need
\usepackage{mathtools, amsfonts, amssymb}
% Links in a pdf
\usepackage{hyperref}
% Use in pictures, graphs, and figures
\usepackage{graphicx}
% Header package
\usepackage{fancyhdr}
% Underlining with line breaks
\usepackage{ulem}
% Adjust accordingly given warning messages
\setlength{\headheight}{15pt}
% So we can more easily format text with pictures
\usepackage{float}
% Images and drawing graphs
\usepackage{tikz}

% Sets footer
\pagestyle{fancy}
% Removes default footer style
\fancyhf{}

\rhead{
  Shengdong Li
  Calc 3
}

\rfoot{
  Page \thepage{}
}

% Makes links look more appealing
\hypersetup{
    colorlinks=true,
    linkcolor=blue,
    filecolor=magenta,      
    urlcolor=cyan,
}

% \usepackage{indentfirst}

\begin{document}
\title{Response to Ankur Moolky}
\author{by Shengdong Li}
\date{23 January 2021}
\maketitle

\begin{align*}
	\intertext{Hello Ankur! Hope you're having a wonderful day. In your initial post, the process that you used to explain every problem was very clear and everything seemed to be correct.}
	\intertext{However, I'm just slightly confused how you solved question 4. You mentioned in order to form a square you needed to rotate the vector 90 degrees 4 times, but I'm still confused as to how exactly you went from the beginning and terminating point of a vector to all four parametric equations of the square. You don't have to go in depth, but could you give a slightly more specific summary of how the square was made?}
	\intertext{In any case, I'll now be turning both randomly generated 2-D space vectors into 3-D vectors, and parametrizing the lines of both!}
	\intertext{The 4 random numbers that I got from Google RNG were}
	& 7,\ 2,\ 4,\ 8                                    \\
	\intertext{Adding these numbers to the initial and terminating points of both of the 2-D vectors, we would get}
	P               & =\left(1,4,7\right)            \\
	Q               & =\left(4,8,2\right)            \\
	R               & =\left(6,8,4\right)            \\
	S               & =\left(4,6,8\right)            \\
	\intertext{In order to get the line between two points, I'll be using the formula }
	r               & =r_{0}+tv                      \\
	\intertext{Where \(r_0\) is a point on the line and \(v\) is the component form of the vector. For the first vector,}
	v               & =<4-1,8-4,2-7>                 \\
	                & =<3,4,-5>                      \\
	\intertext{And we just have to set \(r_0\) to the initial point of the vector}
	r_{0}           & =<1,4,7>                       \\
	\intertext{Plugging these points in, we get}
	r               & =<1,4,7>+t<3,4,-5>             \\
	                & =<3t+1,4t+4,-5t+7>             \\
	\intertext{Thus we can extract the parametric equation of the line from this }
	x               & =3t+1                          \\
	y               & =4t+4                          \\
	z               & =-5t+7                         \\
	r\left(t\right) & =\left(3t+1,4t+4,-5t+7\right)  \\
	\intertext{Now we can basically use the same process for the second line,}
	v               & =<4-6,6-8,8-4>                 \\
	                & =<-2,-2,4>                     \\
	r_{0}           & =<6,8,4>                       \\
	r               & =<6,8,4>+t<-2,-2,4>            \\
	                & =<-2t+6,-2t+8,4t+4>            \\
	x               & =-2t+6                         \\
	y               & =-2t+8                         \\
	z               & =4t+4                          \\
	r\left(t\right) & =\left(-2t+6,-2t+8,4t+4\right) \\
	\intertext{You can check out the two parametric lines below on the embedded GeoGebra graph below! (I threw in the initial and terminal points of the vectors to check as well :P) }
\end{align*}

\end{document}