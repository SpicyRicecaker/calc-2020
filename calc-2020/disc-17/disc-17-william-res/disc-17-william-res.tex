% Font size, paper type
\documentclass[12pt]{article}
% Aesthetic margins
\usepackage[margin=1in]{geometry}
% Core math packages,
% Mathtools loads amsmath, and amsmath gives basic math symbs
% Amsfonts & amssymb are misc. symbols you might need
\usepackage{mathtools, amsfonts, amssymb}
% Links in a pdf
\usepackage{hyperref}
% Use in pictures, graphs, and figures
\usepackage{graphicx}
% Header package
\usepackage{fancyhdr}
% Underlining with line breaks
\usepackage{ulem}
% Adjust accordingly given warning messages
\setlength{\headheight}{15pt}
% So we can more easily format text with pictures
\usepackage{float}
% Images and drawing graphs
\usepackage{tikz}

% Sets footer
\pagestyle{fancy}
% Removes default footer style
\fancyhf{}

\rhead{
  Shengdong Li
  Calc 3
}

\rfoot{
  Page \thepage{}
}

% Makes links look more appealing
\hypersetup{
    colorlinks=true,
    linkcolor=blue,
    filecolor=magenta,      
    urlcolor=cyan,
}

% \usepackage{indentfirst}

\begin{document}
\title{Response to William Zhang}
\author{by Shengdong Li}
\date{24 January 2021}
\maketitle
\begin{align*}
\intertext{Hello William! Thank you so much for responding to my initial post. When parametrizing the line between two vectors of another classmate, Joshua's post, I used the equation}
r&=r_{0}+tv\\
\intertext{Where \(r_0\) is the initial point and \(v\) is the component form of the vector}
\intertext{The way that you multiplied the \(t\) straight into the component form of the vector between the other two vectors before adding the initial point \(r_0\) reminded me that not adding \(r_0\), and just using \(tv\) would just result in the component form of the line between the two vectors.}
\intertext{Other than that, everything else seems to line up!}
\end{align*}

\end{document}