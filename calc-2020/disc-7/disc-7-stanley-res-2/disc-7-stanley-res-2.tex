\documentclass[12pt]{article}
\usepackage[margin=1in]{geometry}
\usepackage{amsfonts, amsmath, amssymb}
\usepackage{hyperref}
\usepackage{mathtools}
\hypersetup{
    colorlinks=true,
    linkcolor=blue,
    filecolor=magenta,      
    urlcolor=cyan,
}
\usepackage{graphicx}
\usepackage{fancyhdr}
\setlength{\headheight}{15pt}

\pagestyle{fancy}
\fancyhf{}

\rhead{
  Shengdong Li
  Calc 3
}

\rfoot{
  Page \thepage
}

% \usepackage{indentfirst}

\begin{document}
\title{Response to Stanley Wang 2.0}
\author{by Shengdong Li}
\date{30 October 2020}
\maketitle

\section{Intro}
Hey Stanley!
Verifying your initial post, almost all of our answers were correct, except for the time until $1$ gram is left calculated using sequence formula being $38.2193$ minutes instead of $33.22$ minutes, which you already acknowledged in a comment to my post.

\section{Regarding \texorpdfstring{$a_{n}=\frac{100}{2^{\frac{n}{5}-1}}$}{Lg}}
I think that 
I pointed out earlier in my response to your response to my post, that I thought that $\frac{n}{5}$ is invalid in a sequence. However, now, after some research, I rescind that and say that I think my setup of $n=\frac{t}{5}$ is worse, and your setup is better!
\subsection{Wikipedia def}
If we look at the wikipedia definition for a \href{https://en.wikipedia.org/wiki/Sequence}{sequence}, it says that ``Formally, a sequence can be defined as a function whose domain is either the \textbf{set of the natural numbers} (for infinite sequences), or the \textbf{set of the first n natural numbers} (for a sequence of finite length n).''

In other words, the indices $i$ in the terms $a_i$ of a sequence must have a domain $i\in\mathbb{N}$.

I do remember learning that natural numbers don't include $0$, but again after consulting Wikipedia's definition for \href{https://en.wikipedia.org/wiki/Natural_number}{natural numbers} I learned that it can be either $\{1, 2, 3, 4, ...\}$ in math theory or $\{0, 1, 2, 3, 4, ...\}$ in computer science.
\subsection{Ms. McNamee def}
If we look at the thursday lesson and notes (10/22/20) for last week, we can also see that Ms. McNamee defined the domain as the following: \par
$\{a_n\}=a_1, a_2, a_3, a_4,...$\par
all $a_i$ are called "terms", $i\in\mathbb{Z}\:st\:i>0\:\text{(usually)}$

\section{Application}
Therefore, I think that considering that sequences are intended to have a domain of $\mathbb{N}$, your formula works better than mine because you would be able to plugin numbers at 3 minutes, 7 minutes, and other minutes that are not a multiple of 5 into your formula of $a_{n}=\frac{100}{2^{\frac{n}{5}-1}}$ because $\{3,7\}\in\mathbb{N}$. 

Meanwhile, for me, I would be unable to plugin a number like 3 minutes, into my formula of $a_{n}=\frac{100}{2^{n}}$ because I would have to calculate half-lives first:
\begin{align}
n&=\frac{t}{5}\\
n&=\frac{3}{5}\\
n&=0.6
\end{align}
And $\{0.6\}\not\in\mathbb{N}$ 

\section{Implication}
Finally, as you mentioned, if you directly are substituting in minutes to $n$ then I think it would make more sense to remove the $-1$ in $a_{n}=\frac{100}{2^{\frac{n}{5}-1}}$, otherwise you would have to add $5$ minutes to every $n$ value as you input, as you mentioned earlier in the response to my response to your response to my post.

\section{In Conclusion}
\underline{Bottom line is, $a_{n}=\frac{100}{2^{\frac{n}{5}}}$ is better than $a_{n}=\frac{100}{2^{n}}$.}
Sorry for causing any confusion, and thank you for your patience!

Cheers, 

Andy
\end{document}