\documentclass[12pt]{article}
\usepackage[margin=1in]{geometry}
\usepackage{amsfonts, amsmath, amssymb}
\usepackage{hyperref}
\usepackage{mathtools}
\hypersetup{
    colorlinks=true,
    linkcolor=blue,
    filecolor=magenta,      
    urlcolor=cyan,
}
\usepackage{graphicx}
\usepackage{fancyhdr}
\setlength{\headheight}{15pt}

\pagestyle{fancy}
\fancyhf{}

\rhead{
  Shengdong Li
  Calc 3
}

\rfoot{
  Page \thepage
}

% \usepackage{indentfirst}

\begin{document}
\title{Response to Stanley Wang}
\author{by Shengdong Li}
\date{29 October 2020}
\maketitle
Hey Stanley! Thank you so much for responding to my initial post!
It's nice to see that you verified my formulae as accurate.

Examining your initial sequence formula for amount of substance left at $a_{n}=\frac{100}{2^{n-1}}$, I can see that it differed from my initial formula of $a_{n}=100\cdot\left(\frac{1}{2}\right)^{n}$. First, your formula was much more simplified than mine. In my initial, I wrote my formula as is because I thought that it would be better in the form of an exponential equation, of $f\left(x\right)=a\cdot\left(b\right)^{r}$. Now, seeing your equation, I realize that $a_{n}=\frac{100}{2^{n}}$ is much less expensive to calculate. The other difference was that you assumed that the sequence would start at $a_{1}$ and thus added the $n-1$ to account for it while I assumed that the sequence would start at $a_{0}$ and thus kept my $n$ the same.

However, I disagree with you in that I don't think your starting formula was more confusing than mine. I think that assuming the sequence starts at $0$ or $1$ are both valid options! Instead of defining $n=\frac{t}{5}$, you would just have to define it as $n-1=\frac{t}{5}$. Then, if you were solving for the amount of substance left at $100$ minutes, you would just compute the following for $n$
\begin{align}
  \frac{100}{5} & =n-1 \\
  20            & =n-1 \\
  \Aboxed{21    & =n}
\end{align}
Therefore you would still end up with $n=21$, which you could plugin to the formula to get an equal answer.

So I think that the reason you may have thought that your formula was more complicated was because of your second formula of $a_{n}=\frac{100}{2^{\frac{n}{5}-1}}$, which I think may be slightly incorrect based on my understanding of how incrementing in sequences work
(more on that in a comment to your post).

Again, thanks for replying!

Cheers,
Andy
\end{document}