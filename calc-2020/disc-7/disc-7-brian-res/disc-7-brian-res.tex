% Font size, paper type
\documentclass[12pt]{article}
% Aesthetic margins
\usepackage[margin=1in]{geometry}
% Core math packages,
% Mathtools loads amsmath, and amsmath gives basic math symbs
% Amsfonts & amssymb are misc. symbols you might need
\usepackage{mathtools, amsfonts, amssymb}
% Links in a pdf
\usepackage{hyperref}
% Use in pictures, graphs, and figures
\usepackage{graphicx}
% Header package
\usepackage{fancyhdr}
% Underlining with line breaks
\usepackage{ulem}
% Adjust accordingly given warning messages
\setlength{\headheight}{15pt}

% Sets footer
\pagestyle{fancy}
% Removes default footer style
\fancyhf{}

\rhead{
  Shengdong Li
  Calc 3
}

\rfoot{
  Page \thepage
}

% Makes links look more appealling
\hypersetup{
    colorlinks=true,
    linkcolor=blue,
    filecolor=magenta,      
    urlcolor=cyan,
}

% \usepackage{indentfirst}

\begin{document}
\title{Response to Brian Chou}
\author{by Shengdong Li}
\date{31 October 2020}
\maketitle

\section{Intro}

Hello Brian,

In your comment to Jackie's initial post, I disagree with your point that Jackie's formula of $a_{n}=\frac{100}{2^{n}}$ is better than $a_{n}=\frac{100}{2^{\frac{n}{5}}}$.

\section{Sequence def}
I think this because in my previous comment to Stanley Wang, I relearned that sequences only actually have a domain of natural numbers.

In other words, the indices $i$ in the terms $a_i$ of a sequence must have a domain $i\in\mathbb{N}$, and $\mathbb{N}$ could be $\{0, 1, 2, 3, 4,...\}$ or $\{1,2,3,4,...\}$ depending on the context.

You can verify this info with the definition of \href{https://en.wikipedia.org/wiki/Sequence}{sequences} on wikipedia, or look at Ms. McNamee's notes on sequences (week 6, thursday, 10/22/20), though she defined the domain of sequences as positive integers.

\section{Application}
This effectively means, that if you wanted to find the amount of substance left at say, 11 minutes, for the formula $a_{n}=\frac{100}{2^{\frac{n}{5}}}$ you would be able to plugin 11 as $n$ straight into the formula because it is a positive integer and natural number and thus exists on the sequence. 

However, for the formula $a_{n}=\frac{100}{2^{n}}$ you would first have to calculate the amount of half-lives by dividing those 11 minutes by the half life of 5 minutes, which results in 2.2. \uline{However, because 2.2 is not a natural number or a positive integer, you cannot plug it into the sequence formula because that number \textit{doesn't exist} on the sequence.}

This may be one of the weaknesses of using sequences over differential equations, because of the more limiting domain.

Thanks for reading,

Andy
\end{document}