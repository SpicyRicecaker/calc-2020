% Font size, paper type
\documentclass[12pt]{article}
% Aesthetic margins
\usepackage[margin=1in]{geometry}
% Core math packages,
% Mathtools loads amsmath, and amsmath gives basic math symbs
% Amsfonts & amssymb are misc. symbols you might need
\usepackage{mathtools, amsfonts, amssymb}
% Links in a pdf
\usepackage{hyperref}
% Use in pictures, graphs, and figures
\usepackage{graphicx}
% Header package
\usepackage{fancyhdr}
% Underlining with line breaks
\usepackage{ulem}
% Adjust accordingly given warning messages
\setlength{\headheight}{15pt}

% Sets footer
\pagestyle{fancy}
% Removes default footer style
\fancyhf{}

\rhead{
  Shengdong Li
  Calc 3
}

\rfoot{
  Page \thepage
}

% Makes links look more appealling
\hypersetup{
    colorlinks=true,
    linkcolor=blue,
    filecolor=magenta,      
    urlcolor=cyan,
}

% \usepackage{indentfirst}

\begin{document}
\title{Response to Omar Nassar}
\author{by Shengdong Li}
\date{6 November 2020}
\maketitle


\section{Intro}

Hello Omar, really interesting initial post!

The series ratio test seems like a test that makes a lot of sense: if the ratio between the two terms is greater than 1, then the series is convergent. I don't really get how the ratio of 1 may diverge or converge though, so I'm looking forward to learning that later!

\section{Divergency Test}
Another more simpler way that you could approach the problem is just to apply the divergency test first. Below I'll show my response to Jyotsna, and I think it's a lot easier than the ratio test.

\begin{align}
\lim_{n\to\infty}\frac{6^{n}}{5^{n}-1}
\intertext{Next we can multiple the top and bottom by $\frac{1}{5^n}$}
&=\lim_{n\to\infty}\frac{\frac{6^{n}}{5^{n}}}{1-\frac{1}{5^{n}}}
\intertext{We can use the exponent rules to take $n$ out of the numerator}
&=\lim_{n\to\infty}\frac{\left(\frac{6}{5}\right)^{n}}{1-\frac{1}{5^{n}}}
\intertext{Evaluating this we get}
&=\frac{\infty}{1-0}\\
&=\infty\\
&=\boxed{diverges}
\end{align}
Now because the $\lim_{n\to\infty}\frac{6^{n}}{5^{n}-1}$ diverges, then by the divergency test we know that the series must also converge.

I hope that helps!

\bigskip

Cheers,

- Andy

\end{document}