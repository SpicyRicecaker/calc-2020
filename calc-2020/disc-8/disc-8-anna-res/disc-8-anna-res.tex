% Font size, paper type
\documentclass[12pt]{article}
% Aesthetic margins
\usepackage[margin=1in]{geometry}
% Core math packages,
% Mathtools loads amsmath, and amsmath gives basic math symbs
% Amsfonts & amssymb are misc. symbols you might need
\usepackage{mathtools, amsfonts, amssymb}
% Links in a pdf
\usepackage{hyperref}
% Use in pictures, graphs, and figures
\usepackage{graphicx}
% Header package
\usepackage{fancyhdr}
% Underlining with line breaks
\usepackage{ulem}
% Adjust accordingly given warning messages
\setlength{\headheight}{15pt}

% Sets footer
\pagestyle{fancy}
% Removes default footer style
\fancyhf{}

\rhead{
  Shengdong Li
  Calc 3
}

\rfoot{
  Page \thepage
}

% Makes links look more appealling
\hypersetup{
    colorlinks=true,
    linkcolor=blue,
    filecolor=magenta,      
    urlcolor=cyan,
}

% \usepackage{indentfirst}

\begin{document}
\title{Response to Anna Jiang}
\author{by Shengdong Li}
\date{6 November 2020}
\maketitle
Hello Anna, thank you so much for responding to my initial post!

Your suggestion about first using the convergence test to see if the series of $\sum_{n=1}^{\infty}\frac{\sin\left(\frac{1}{k}\right)}{k^{2}}$ has the potential to converge or not is a really great suggestion.  I realize now that I often gauge the ``possibility that a series may converge'' before I begin proving it using one of the convergence tests anyway. For this problem, it was obvious to see that $\sin\left(\frac{1}{k}\right)$ of something over $k^{2}$ would probably converge. But writing out $\lim_{k\to\infty}\frac{\sin\left(\frac{1}{k}\right)}{k^{2}}$ and solving it definitely shows my thought process better and can help a lot, especially for hard problems. 

Your method in solving this problem using the limit comparison test with $a_{n}=\frac{\sin\left(\frac{1}{k}\right)}{k^{2}}$ and $b_{n}=\frac{1}{k^{2}}$ also works very well.

Again, thanks for responding to my initial post!

- Andy

\end{document}