% Font size, paper type
\documentclass[12pt]{article}
% Aesthetic margins
\usepackage[margin=1in]{geometry}
% Core math packages,
% Mathtools loads amsmath, and amsmath gives basic math symbs
% Amsfonts & amssymb are misc. symbols you might need
\usepackage{mathtools, amsfonts, amssymb}
% Links in a pdf
\usepackage{hyperref}
% Use in pictures, graphs, and figures
\usepackage{graphicx}
% Header package
\usepackage{fancyhdr}
% Underlining with line breaks
\usepackage{ulem}
% Adjust accordingly given warning messages
\setlength{\headheight}{15pt}

% Sets footer
\pagestyle{fancy}
% Removes default footer style
\fancyhf{}

\rhead{
  Shengdong Li
  Calc 3
}

\rfoot{
  Page \thepage
}

% Makes links look more appealling
\hypersetup{
    colorlinks=true,
    linkcolor=blue,
    filecolor=magenta,      
    urlcolor=cyan,
}

% \usepackage{indentfirst}

\begin{document}
\title{In Conclusion}
\author{by Shengdong Li}
\date{8 November 2020}
\maketitle

\section{Intro}

In this discussion I reaffirmed my understanding of some of the methods that one could use to approach testing if a series is convergent or not, and was able to practice and notate the different ways of showing the process. 

\section{Methods Used}
There were three main methods used to test the covergency of $\sum_{n=1}^{\infty} \frac{\sin\left(\frac{1}{k}\right)}{k^{2}}$. These were
\begin{itemize}
  \item Direct Comparison Test with $a_n=\sum_{k=1}^{\infty}\frac{\sin\left(\frac{1}{k}\right)}{k^{2}}$ and $b_n=\sum_{k=1}^{\infty}\frac{1}{k^{2}}$ (mine)
  \item Limit Comparison Test with $a_n=\sum_{k=1}^{\infty}\frac{\sin\left(\frac{1}{k}\right)}{k^{2}}$ and $b_n=\sum_{k=1}^{\infty}\frac{1}{k^{2}}$ (Anna Jiang)
  \item Integral Test with $u=\frac{1}{k}$ (Amuthan Llavarasan)
\end{itemize}

\subsection{Best Method}

In my opinion, the best of these tests was probably the limit comparison test that Anna used. Because in $\frac{b_k}{a_k}$ the $k^2$s cancel, all you had to do was simply compute the limit of $\sin(1/k)$ as $n$ approached infinity. This was very simple and required the least amount of steps. I think that the direct comparison test was also a good contender, but required the evaluation of a $p$-series and proof that $b_n$ was greater than $a_n$. Finally, I think that the integral test by Amuthan was also a solid approach and was very straightforward to solve, but in the end required the most amount of steps.

\section{Divergency Test}
Anna Jiang's comment to my initial post about first using the divergency test when testing for the convergency of series is a really good piece of advice. Because I hadn't used it formally for so long, I almost forgot about it. Thanks to her comment I was reminded about how it worked, and realized that another approach to the problem $\frac{6^{n}}{5^{n}-1}$ could be straight up taking the limit of it as $n$ approached infinity, which allowed me to spread the knowledge and comment to many other people's posts.

\end{document}