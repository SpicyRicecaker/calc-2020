% Font size, paper type
\documentclass[12pt]{article}
% Aesthetic margins
\usepackage[margin=1in]{geometry}
% Core math packages,
% Mathtools loads amsmath, and amsmath gives basic math symbs
% Amsfonts & amssymb are misc. symbols you might need
\usepackage{mathtools, amsfonts, amssymb}
% Links in a pdf
\usepackage{hyperref}
% Use in pictures, graphs, and figures
\usepackage{graphicx}
% Header package
\usepackage{fancyhdr}
% Underlining with line breaks
\usepackage{ulem}
% Adjust accordingly given warning messages
\setlength{\headheight}{15pt}

% Sets footer
\pagestyle{fancy}
% Removes default footer style
\fancyhf{}

\rhead{
  Shengdong Li
  Calc 3
}

\rfoot{
  Page \thepage
}

% Makes links look more appealling
\hypersetup{
    colorlinks=true,
    linkcolor=blue,
    filecolor=magenta,      
    urlcolor=cyan,
}

% \usepackage{indentfirst}

\begin{document}
\title{\texorpdfstring{$\frac{1}{1+x}$}{Lg}}
\author{by Shengdong Li}
\date{21 November 2020}
\maketitle

Stefano, our group member assigned to find the maclaurin series for
question 5
$$
  \frac{1}{1+x}
$$
hasn't posted. It's now saturday, and since I had question 4, to find the maclaurin series for
$$
  \frac{1}{1-x}
$$
I decided to work through his question so I could try and compare these two.

\section{\texorpdfstring{$\frac{1}{1+x}$}{Lg}}
To find the first five terms of the maclaurin series for $\frac{1}{1-x}$, we have to find the first 5 derivatives of $f(x)$
\begin{align}
  f\left(x\right)                  & =\left(1+x\right)^{-1}     \\
  f'\left(x\right)                 & =-\left(1+x\right)^{-2}    \\
  f''\left(x\right)                & =2\left(1+x\right)^{-3}    \\
  f'''\left(x\right)               & =-6\left(1+x\right)^{-4}   \\
  f^{\left(4\right)}\left(x\right) & =24\left(1+x\right)^{-5}   \\
  f^{\left(5\right)}\left(x\right) & =-120\left(1+x\right)^{-6} \\
  \intertext{We know that in a maclaurin series the function is centered at $a=0$. We plugin $0$ into our previously calculated derivatives}
  f\left(0\right)                  & =1                         \\
  f'\left(0\right)                 & =-1                        \\
  f''\left(0\right)                & =2                         \\
  f'''\left(0\right)               & =-6                        \\
  f^{\left(4\right)}\left(0\right) & =24                        \\
  f^{\left(5\right)}\left(0\right) & =-120                      \\
\end{align}
We can plug these derivatives into the general formula for taylor polynomials, and write out the expanded form
\begin{align}
  \frac{1}{1+x} & =1-x+\frac{2x^{2}}{2!}-\frac{6x^{3}}{3!}+\frac{24x^{4}}{4!}-\frac{120x^{5}}{5!}+\ldots \\
  \Aboxed{{}    & =1-x+x^{2}-x^{3}+x^{4}-x^{5}+\ldots}
\end{align}
Which we could write in summation form as
\begin{align}
             & =\sum_{n=0}^{\infty}\left(-1\right)^{n}x^{n} \\
  \Aboxed{{} & =\sum_{n=0}^{\infty}\left(-x\right)^{n}}
\end{align}
Another way to approach finding the summation form of the series is to define $f^{(n)}(x)$. From the derivatives above, we can see that the $n$th derivative $f^{(n)}(x)$ would be alternating $n!$
\begin{align}
  f^{\left(n\right)}\left(x\right) & =(-1)^n\cdot n!                                                                       \\
  \frac{1}{1+x}                    & = \sum_{n=0}^{\infty}\frac{f^{\left(n\right)}\left(x\right)\cdot\left(x-a\right)}{n!} \\
  \frac{1}{1+x}                    & = \sum_{n=0}^{\infty}\frac{(-1)^n\cdot n!\cdot x}{n!}                                 \\
\end{align}
Plugging into the formula, the $n!$s cancel, leaving
\begin{align}
             & = \sum_{n=0}^{\infty}(-1)^n\cdot x       \\
  \Aboxed{{} & =\sum_{n=0}^{\infty}\left(-x\right)^{n}}
\end{align}
\section{\texorpdfstring{$\frac{1}{1+2x}$}{Lg}}
To find the first five terms of the maclaurin series for $\left(1-2x\right)^{-1}$, we have to find the first 5 derivatives of $f(x)$
\begin{align}
  f\left(x\right)                  & =\left(1-2x\right)^{-1}      \\
  f'\left(x\right)                 & =-2\left(1-2x\right)^{-2}    \\
  f''\left(x\right)                & =8\left(1-2x\right)^{-3}     \\
  f'''\left(x\right)               & =-48\left(1-2x\right)^{-4}   \\
  f^{\left(4\right)}\left(x\right) & =384\left(1-2x\right)^{-5}   \\
  f^{\left(5\right)}\left(x\right) & =-3840\left(1-2x\right)^{-6} \\
  \intertext{Plugin $x=0$, as maclaurin series are centered at $0$}
  f\left(0\right)                  & =1                           \\
  f'\left(0\right)                 & =-2                          \\
  f''\left(0\right)                & =8                           \\
  f'''\left(0\right)               & =-48                         \\
  f^{\left(4\right)}\left(0\right) & =384                         \\
  f^{\left(5\right)}\left(0\right) & =-3840
\end{align}
\begin{align}
  \intertext{Then we can plug in the results into the taylor formula to get the expanded form with the first $5$ terms}
  \frac{1}{1+x} & =1-2x+\frac{8x^{2}}{2!}-\frac{48x^{3}}{3!}+\frac{384x^{4}}{4!}-\frac{3840x^{5}}{5!}+\ldots \\
  \Aboxed{{}    & =1-2x+4x^{2}-8x^{3}+16x^{4}-32x^{5}+\ldots}
  \intertext{And in summation form}
                & = \sum_{n=0}^{\infty}\left(-1\right)^{n}\cdot2^{n}\cdot x^{n}                              \\
  \Aboxed{   {} & = \sum_{n=0}^{\infty}\left(-2x\right)^{n}}
\end{align}
\begin{align}
  \intertext{Another way we could approach finding the summation is to find $f^{(n)}(x)$ and plug that directly into the formula}
  \frac{1}{1+x}                    & =\frac{f^{\left(n\right)}\left(x\right)\cdot\left(x-a\right)}{n!}  \\
  f^{\left(n\right)}\left(x\right) & =(-1)^n\cdot n!\cdot2^{n}                                          \\
                                   & =\sum_{n=0}^{\infty}\frac{(-1)^n\cdot n!\cdot2^{n}\cdot x^{n}}{n!} \\
                                   & =\sum_{n=0}^{\infty}\left(-1\right)^{n}\cdot2^{n}                  \\
  \Aboxed{ {}                      & =\sum_{n=0}^{\infty}\left(-2x\right)^{n}}                          \\
\end{align}
\section{\texorpdfstring{$\frac{1}{1+x^{2}}$}{Lg}}
To find the first five terms, find the first $5$ derivatives
\begin{align}
  f\left(x\right)                  & =\frac{1}{1+x^{2}}                                                   \\
  f\left(x\right)                  & =\left(1+x^{2}\right)^{-1}                                           \\
  f'\left(x\right)                 & =-2x\left(1+x^{2}\right)^{-2}                                        \\
  f''\left(x\right)                & =-\frac{2\left(1-3x^{2}\right)}{\left(1+x^{2}\right)^{3}}            \\
  f'''\left(x\right)               & =24x\left(\frac{1-x^{2}}{\left(1+x^{2}\right)^{4}}\right)            \\
  f^{\left(4\right)}\left(x\right) & =24\left(\frac{5x^{4}-10x^{2}+1}{\left(1+x^{2}\right)^{5}}\right)    \\
  f^{\left(5\right)}\left(x\right) & =240x\left(\frac{-3x^{4}+10x^{2}-3}{\left(1+x^{2}\right)^{6}}\right) \\
  \intertext{Next, plugin $x=0$}
  f\left(0\right)                  & =1                                                                   \\
  f'\left(0\right)                 & =0                                                                   \\
  f''\left(0\right)                & =-2                                                                  \\
  f'''\left(0\right)               & =0                                                                   \\
  f^{\left(4\right)}\left(0\right) & =24                                                                  \\
  f^{\left(5\right)}\left(0\right) & =0                                                                   \\
  \intertext{Plugging the results into the taylor formula can give us the expanded form}
  \frac{1}{1+x^2}                  & =1-\frac{2x^{2}}{2!}+\frac{24x^{4}}{4!}-\ldots                       \\
  \Aboxed{{}                       & =1-x^{2}+x^{4}-\ldots}
  \intertext{And summation form}
                                   & =\sum_{n=0}^{\infty}\left(-1\right)^{n}x^{2n}                        \\
  \Aboxed{                         & =\sum_{n=0}^{\infty}\left(-x^{2}\right)^{n}}
\end{align}

\section{My observations}
Immediately, what stood out to me between the maclaurin series for $\frac{1}{1-x}$ and $\frac{1}{1+x}$ is that one could literally drop in the $\pm x$ and get the summation form of the taylor series for each, as shown below.
$$\sum_{n=0}^{\infty}\left(x\right)^{n}$$
$$\sum_{n=0}^{\infty}\left(-x\right)^{n}$$
$$\sum_{n=0}^{\infty}\left(2x\right)^{n}$$
$$\sum_{n=0}^{\infty}\left(-2x\right)^{n}$$
$$\sum_{n=0}^{\infty}\left(x^{2}\right)^{n}$$
$$\sum_{n=0}^{\infty}\left(-x^{2}\right)^{n}$$
In addition, looking at the above differences the summation form for getting from $x$ to $x^2$ and $x$ to $2x$ for both $\frac{1}{1-x}$ and $\frac{1}{1+x}$ were $also$ drop in replacements...

This leads me to believe that perhaps a possible shortcut for finding the taylor series of functions with similar structures is to look at their $x$s or other variables, and drop in the corresponding $x$ accordingly.
\end{document}