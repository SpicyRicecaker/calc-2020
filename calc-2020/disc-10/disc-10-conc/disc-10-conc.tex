% Font size, paper type
\documentclass[12pt]{article}
% Aesthetic margins
\usepackage[margin=1in]{geometry}
% Core math packages,
% Mathtools loads amsmath, and amsmath gives basic math symbs
% Amsfonts & amssymb are misc. symbols you might need
\usepackage{mathtools, amsfonts, amssymb}
% Links in a pdf
\usepackage{hyperref}
% Use in pictures, graphs, and figures
\usepackage{graphicx}
% Header package
\usepackage{fancyhdr}
% Underlining with line breaks
\usepackage{ulem}
% Adjust accordingly given warning messages
\setlength{\headheight}{15pt}

% Sets footer
\pagestyle{fancy}
% Removes default footer style
\fancyhf{}

\rhead{
  Shengdong Li
  Calc 3
}

\rfoot{
  Page \thepage
}

% Makes links look more appealling
\hypersetup{
    colorlinks=true,
    linkcolor=blue,
    filecolor=magenta,      
    urlcolor=cyan,
}

% \usepackage{indentfirst}

\begin{document}
\title{In Conclusion}
\author{by Shengdong Li}
\date{22 November 2020}
\maketitle

\section{Taylor Polynomial vs. Taylor Series}
In Michael Yan's reply to my initial post, I was reminded to simplify the expanded form of the maclaurin series for my assigned functionm, but also questioned about not using $T_{n}(x)$. This lead to me realizing that a taylor polynomial is the partial sum of the derivatives of a function while a taylor series is the infinite sum of the derivatives of a function.

\section{Shortcuts}
From my initial post, comparing taylor series between $g(x)=x$, $g(x)=2x$ and $g(x)=x^2$, I think that one possible shortcut to solving the taylor series of $f(g(x))$ already knowing $f(x)$ is just to substitute $g(x)$ into the summation formula.

Comparing the differences between $\frac{1}{1-x}$ and $\frac{1}{1+x}$ also confirms the aforementioned theory, as substituting $(-x)$ for $x$ in the taylor series for $\frac{1}{1-x}$ gives the taylor series for $\frac{1}{1+x}$.

Finally, from my comment to William's initial post noting the differences between the maclaurin series for $\sin(x)$ and $\cos(x)$, I realized that a shortcut could be taking the derivative of the taylor series of one function to get to the other.

\section{\texorpdfstring{$\frac{x}{1-x}$}{Lg}}
\begin{align}
  \intertext{The first five derivatives}
  f\left(x\right)                  & =\frac{x}{1-x}                                                                      \\
  f'\left(x\right)                 & =\frac{1}{\left(1-x\right)^{2}}                                                     \\
  f''\left(x\right)                & =\frac{2}{\left(1-x\right)^{3}}                                                     \\
  f'''\left(x\right)               & =\frac{6}{\left(1-x\right)^{4}}                                                     \\
  f^{\left(4\right)}\left(x\right) & =\frac{24}{\left(1-x\right)^{5}}                                                    \\
  f^{\left(5\right)}\left(x\right) & =\frac{120}{\left(1-x\right)^{6}}                                                   \\
\end{align}
Substitute $x=0$
\begin{align}
  f\left(0\right)                  & =0                                                                                  \\
  f'\left(0\right)                 & =1                                                                                  \\
  f''\left(0\right)                & =2                                                                                  \\
  f'''\left(0\right)               & =6                                                                                  \\
  f^{\left(4\right)}\left(0\right) & =24                                                                                 \\
  f^{\left(5\right)}\left(0\right) & =120                                                                                
\end{align}
\begin{align}
  \intertext{expanded form}
  \frac{x}{1-x}                    & =0+x+\frac{2x^{2}}{2!}+\frac{6x^{3}}{3!}+\frac{24x^{4}}{4!}+\frac{120x^{5}}{5!}+... \\
  \Aboxed{\frac{x}{1-x}                    & =x+x^{2}+x^{3}+x^{4}+x^{5}+...}
\end{align}
\begin{align}
  \intertext{evaluating it directly, we can see that the expanded form is the same as $\frac{1}{1-x}$, just ``shifted over'' by 1, hence the $+1$}
  \Aboxed{\frac{x}{1-x}                    & =\sum_{n=0}^{\infty}x^{n+1}}
\end{align}
\section{Shortcut}
\begin{align}
  \intertext{because in the context of the summation, $x$ doesn't change, the shortcut would probably be to just multiply it in}
  \frac{1}{1-x}                    & =\sum_{n=0}^{\infty}x^{n}                                                           \\
  \frac{x}{1-x}                    & =\sum_{n=0}^{\infty}x^{n}\cdot x                                                    \\
  \frac{x}{1-x}                    & =\sum_{n=0}^{\infty}x^{n+1}                                                         
\end{align}
\end{document}