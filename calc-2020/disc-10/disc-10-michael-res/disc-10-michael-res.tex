% Font size, paper type
\documentclass[12pt]{article}
% Aesthetic margins
\usepackage[margin=1in]{geometry}
% Core math packages,
% Mathtools loads amsmath, and amsmath gives basic math symbs
% Amsfonts & amssymb are misc. symbols you might need
\usepackage{mathtools, amsfonts, amssymb}
% Links in a pdf
\usepackage{hyperref}
% Use in pictures, graphs, and figures
\usepackage{graphicx}
% Header package
\usepackage{fancyhdr}
% Underlining with line breaks
\usepackage{ulem}
% Adjust accordingly given warning messages
\setlength{\headheight}{15pt}

% Sets footer
\pagestyle{fancy}
% Removes default footer style
\fancyhf{}

\rhead{
  Shengdong Li
  Calc 3
}

\rfoot{
  Page \thepage
}

% Makes links look more appealling
\hypersetup{
    colorlinks=true,
    linkcolor=blue,
    filecolor=magenta,      
    urlcolor=cyan,
}

% \usepackage{indentfirst}

\begin{document}
\title{Response to Michael Yan}
\author{by Shengdong Li}
\date{20 November 2020}
\maketitle

Hello Michael,
Thank you so much for responding to my initial post!

\section{Simplifying the Expanded Form}
I definitely see your comments about my simplification. I should've simplified my expanded form of the first 5 terms more, instead of moving straight into the summation form.

Rewriting the expanded forms for each question respectively gave me
$$
  \frac{1}{1-x}=1+x+x^2+x^3+x^4+x^5+...
$$
$$
  \frac{1}{1-2x}=1+2x+4x^2+8x^3+16x^4+x^5+...
$$
$$
  \frac{1}{1-x^2}=1+x^2+x^4...
$$
And these were the same that you got listed in your comment!

\section{Taylor Polynomial vs. Taylor Series}
Your other comment on my post was about the notation of the expanded form that I used. I'm going to be honest, this made me doubt my notation greatly. However, after heavy research, I think that I've come to a conclusion about the differences between taylor series and taylor polynomials, and justified my notation. Please take a look below...

\subsection{How I wrote it in the beginning}
I wrote my expanded forms with $f(x)$ on the left and the terms on the right with ellipses like so
$$
  \frac{1}{1-x}=1+x+\frac{2x^{2}}{2!}+\frac{6x^{3}}{3!}+\frac{24x^{4}}{4!}+\frac{120x^{5}}{5!}+\ldots
$$

\subsection{Reasoning After Research}
If we look at the prompt, it states ``For whichever problem you were assigned, first determine the \uline{Maclaurin Series} for each writing it in summation form and \uline{in its expanded form for the first 5 terms}''.

According to the \href{https://en.wikipedia.org/wiki/Taylor_series}{Wikipedia article}, ``the \textbf{Taylor series} of a function is an \uline{infinite sum of terms} that are expressed in terms of the function's derivatives at a certain point.''
that is, the summation notation would go to $\infty$
$$
  \sum_{n=0}^{\infty}
$$
While on the other hand, a \textbf{Taylor polynomial} is ``\uline{the partial sum} formed by the n first terms of a Taylor series is a polynomial of degree n ''
that is, the summation notation would only go until $n$
$$
  \sum_{n=0}^{n}
$$

Therefore, the ``Maclaurin Series'' would be the full sum of terms to infinity, and not just the partial sum of terms resulting in the taylor polynomial of the 5th degree $T_{5}(x)$. 

Based off this, I concluded that $T_{5}(x)$ and $f(x)$ were not interchangeable because
$$5\neq\infty$$
and I believe this is reflected in everyone's summation form in the discussion
$$
  \sum_{n=0}^{5}\frac{f^{\left(n\right)}\left(x\right)\cdot\left(x-a\right)}{n!}\neq\sum_{n=0}^{\infty}\frac{f^{\left(n\right)}\left(x\right)\cdot\left(x-a\right)}{n!}
$$
where we all put to $\infty$ instead of $5$.

\section{End}
The online book \href{https://mathbooks.unl.edu/Calculus/sec-7-6-taylor.html}{Coordinated Calculus} also had two equations of taylor polynomials and taylor series that helped me understand their differences.

Thanks again for replying, and please respond if you disagree or have any questions. I'm still shaky on this topic myself!
\bigbreak
Cheers,

- Andy
\end{document}