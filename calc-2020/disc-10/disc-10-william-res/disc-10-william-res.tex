% Font size, paper type
\documentclass[12pt]{article}
% Aesthetic margins
\usepackage[margin=1in]{geometry}
% Core math packages,
% Mathtools loads amsmath, and amsmath gives basic math symbs
% Amsfonts & amssymb are misc. symbols you might need
\usepackage{mathtools, amsfonts, amssymb}
% Links in a pdf
\usepackage{hyperref}
% Use in pictures, graphs, and figures
\usepackage{graphicx}
% Header package
\usepackage{fancyhdr}
% Underlining with line breaks
\usepackage{ulem}
% Adjust accordingly given warning messages
\setlength{\headheight}{15pt}

% Sets footer
\pagestyle{fancy}
% Removes default footer style
\fancyhf{}

\rhead{
  Shengdong Li
  Calc 3
}

\rfoot{
  Page \thepage
}

% Makes links look more appealling
\hypersetup{
    colorlinks=true,
    linkcolor=blue,
    filecolor=magenta,      
    urlcolor=cyan,
}

% \usepackage{indentfirst}

\begin{document}
\title{Response to William Zhang}
\author{by Shengdong Li}
\date{22 November 2020}
\maketitle

Hello William!
I noted that both Kate and Michael mentioned that the difference of $2n$ vs $2n+1$ in the summation notation of the maclaurin series for $\sin(x)$ and $\cos(x)$ made sense. 

Kate commented on how the $2x+1$ made $\cos(x)$ always odd in terms of power while the $2x$ made $\sin(x)$ always even in terms of power. Michael said that this $+1$ difference made sense because $\cos(x)$ was just $\sin(x)$ shifted slightly.

However, the observation that I wanted to make has to do with derivatives, and how that could possibly help us solve future taylor series using shortcuts.

\section{Difference in Derivatives}
Your first five derivatives of $\sin(x)$ were
\begin{align}
  f\left(x\right)                  & =\sin x  \\
  f'\left(x\right)                 & =\cos x  \\
  f''\left(x\right)                & =-\sin x \\
  f'''\left(x\right)               & =-\cos x \\
  f^{\left(4\right)}\left(x\right) & =\sin x  \\
  f^{\left(5\right)}\left(x\right) & =\cos x
\end{align}
While Katherines first five derivatives of $\cos(x)$ were
\begin{align}
  f\left(x\right)                  & =\cos x  \\
  f'\left(x\right)                 & =-\sin x \\
  f''\left(x\right)                & =-\cos x \\
  f'''\left(x\right)               & =\sin x  \\
  f^{\left(4\right)}\left(x\right) & =\cos x  \\
  f^{\left(5\right)}\left(x\right) & =-\sin x
\end{align}
Looking at these derivitives, it immediately occured to me that perhaps this $2n+1$ difference can be explained by the fact that $\cos(x)$ is actually the derivative of $\sin(x)$.

\uline{Taking the derivative of $\sin(x)$ would make it $\cos(x)$, thereby shifting the terms of $\sin(x)$ to make them the exact same as $\cos(x)$.}

Next, looking at the first three terms of $\sin(x)$ that you got, we can see that they were
$$x-\frac{1}{6}x^{3}+\frac{1}{120}x^{5}$$
and for Katherine's $\cos(x)$, we can see that they were
$$1-\frac{1}{2}x^{2}+\frac{1}{24}x^{3}$$

And like the function derivatives, \uline{if we take the derivative of the expanded form of $\sin(x)$, we can see that it also equals the expanded form of $\cos(x)$!}
\begin{align}
  \frac{d}{dx}\left(\sin x\right) & =\frac{d}{dx}\left(x-\frac{1}{6}x^{3}+\frac{1}{120}x^{5}-...\right) \\
  \cos x                          & =1-\frac{3}{6}x^{2}+\frac{5}{120}x^{3}-...                          \\
  \cos x                          & =1-\frac{1}{2}x^{2}+\frac{1}{24}x^{3}-...
\end{align}

And then, since the expanded form is another way to just write the summation form, I wondered if you could straight up take the derivative of the taylor formula of $\sin(x)$ to somehow get it to match with $\cos(x)$.

\begin{align}
  \frac{d}{dx}\left(\sin x\right) & =\frac{d}{dx}\left(\sum_{n=0}^{\infty}\frac{\left(-1\right)^{n}}{\left(2n+1\right)!}\left(x\right)^{2n+1}\right) \\
  \intertext{Here, inside the summation, since you're finding the derivative with respect to $x$, you can just treat the $n$s as constants. }
  \cos x                          & =\sum_{n=0}^{\infty}\frac{\left(-1\right)^{n}}{\left(2n+1\right)!}\left(2n+1\right)\left(x\right)^{2n}           \\
  \intertext{The $2n+1$s cancel here! This leads to the exact same formula for $\cos(x)$ as Katherine got. }
  \cos x                          & =\sum_{n=0}^{\infty}\frac{\left(-1\right)^{n}}{\left(2n\right)!}\left(x\right)^{2n}                              \\
\end{align}

The fact that \uline{taking the derivative of the maclaurin series of $\sin(x)$ gives the maclaurin series of $\cos(x)$} leads me to believe that taking the derivative of a function with a known taylor could be a shortcut to find the taylor series of one that is unknown.

What do you think?
\bigbreak
- Andy
\end{document}