\documentclass[12pt]{article}
\usepackage[margin=1in]{geometry}
\usepackage{amsfonts, amsmath, amssymb}
\usepackage{hyperref}
\usepackage{mathtools}
\hypersetup{
    colorlinks=true,
    linkcolor=blue,
    filecolor=magenta,      
    urlcolor=cyan,
}
\usepackage{graphicx}
\usepackage{fancyhdr}
\setlength{\headheight}{15pt}

\pagestyle{fancy}
\fancyhf{}

\rhead{
  Shengdong Li
  Calc 3
}

\rfoot{
  Page \thepage
}

% \usepackage{indentfirst}

\begin{document}
\title{In Conclusion}
\author{by Shengdong Li}
\date{18 October 2020}
\maketitle

\section{Intro}
The three problems that I did were the initial physics problem, the population problem in response to Kayla, and finally the tank problem in resonse to Aaron.

\section{Physics Problem}
The physics problem that dealt with freefall refined my understanding of equilibrium solutions. Before doing the problem, I barely knew what they were, just that they were the flat lines in the slope field. However, after realizing that they can be used to solve for the $k$ value in a differential equation by setting the $f'$ to $0$, I now have a much better understanding of them. \par
I also realized from this problem that if there's a very hard function that you don't know how to solve for a certain value, graphing the function is always a good last option.
\subsection{Aaron Li}
Aaron's reply to my initial post reminded me that it's important to parse out every question in a word problem and state a clear answer for each one of them. Especially for differential equations, where there are usually many steps to take to solve for a value, for example creating a model for change over time, then solving it for general and particular solution, then plugging in values and using the resulting value to plugin to another equation.\par
In my reply to Aaron, I also realized that gravity can be either interpreted as positive or negative if the problem does not explicitly say it, but depending on which direction you identify as up or down, you'll have to plugin positive and negative distance values accordingly. 
\subsection{Amuthan Llavarasan}
Amuthan's comment to my initial post taught me that it's not always best to put a differential equation in the form of a FOLE and solve it using the formula. Sometimes, for simple equations, using separation of variables is oftentimes faster and takes less steps than putting it in the formula. \par
Amuthan also had an ingenious way to solve for the $h=-\frac{\frac{2916}{9.8}}{e^{\frac{9.8}{54}t}}-54t+\frac{5882916}{9.8}$ by factoring out the $54$ and using that to solve for the root of the equation, which is a method that I'll definitely keep in mind when solving for future differential equations.
\section{Population Problem}
In verification to Kayla's work on the population problem, I realized how important it is to either keep your rounding consistent or to not round your values until you get to the answer. Many of the parts of the problem asking to population could be dozens of people off just because of rounding values to the nearest hundredth.
\section{Tank Problem}
In verification to Aaron's work on the tank problem, I finally understood how to setup the tank differential of $\frac{dX}{dt}=6\cdot10^{3}-\frac{1}{7.39\cdot10^{4}+2.5t}X$, which I didn't quite understand fully the first time I saw it in class.
\section{In Conclusion Conclusion}
While this discussion killed me over and over again with the amount of time it took to complete each problem, in the end I did learn a lot of standards and tips and tricks to approach types of problems.
\end{document}